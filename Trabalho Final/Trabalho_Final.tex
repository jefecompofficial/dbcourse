\documentclass[12pt]{article}

\title{Sistema de Apoio Para Localização \\ de Pessoas Desaparecidas (SALPD)\\
BAN-II}

\author{Professor: \textbf{Jeferson Souza (jefecomp)}}

\begin{document}

\maketitle

\section{Descrição do Trabalho}

O objetivo deste trabalho é modelar uma base de dados relacional de um sistema de apoio para localização de pessoas desaparecidas (SALPD). Além da modelagem da base de dados, é também necessário implementar uma pequena aplicação que permita inserir, remover, atualizar, e consultar os registros na base de dados. Recomenda-se que a implementação do SALPD seja realizada na linguagem de programação Java. Entretanto, cada equipe tem livre arbítrio para escolher outra linguagem de programação, desde que avise previamente. 

Os seguines requisitos (obrigatórios) devem ser considerados na modelagem de dados e implementação do SALPD:

\begin{itemize}
\itemsep 5mm

\item o modelo de dados deve ser relacional;

\item o modelo de dados deve possuir uma entidade que permita registrar um usuário do sistema;

\item o modelo de dados deve possuir uma entidade que permita registrar os diferentes papéis que um usuário do sistema pode assumir (Administrador, Gestor, Agente, Informante, Anônimo);

\item o modelo de dados deve possuir uma entidade que permita registrar informações sobre uma pessoa desaparecida, tais como:  RG, CPF, nome completo, apelido(s), última localização, usuário que inseriu, usuário que fez a última atualização;

\item o modelo de dados deve possuir uma entidade que permita manter um histórico de todas as localizações informadas sobre uma pessoa desaparecida;

\item o modelo de dados deve possuir uma entidade que permita registrar os detalhes (ex: número de telefone) do contato realizado para dar informações (denúncia) sobre pessoas desaparecidas. A localização do telefone que realizou a denúncia deve ser armazenada juntamente com os detalhes do contato. Caso o número de telefone seja de um celular, simular um rastreamento no sistema implementado para armazenar a localização aproximada do telefone no momento do contato. No caso de telefone fixo, a localização é obtida através do endereço de cadastro da operadora telefônica;

\item a interface e acesso ao dados deve ser genérica, ou seja, independente do banco de dados que estiver sendo utilizado;

\item devem ser criadas visualizações (VIEWS) para facilitar consultas mais complexas a base de dados (ex: pessoa desaparecida a menos de 6 meses, com informação de localização fornecida a menos de 7 dias por denúncia de um informante);

\item devem ser criados índices para otimizar a consulta de informações nas diferentes tabelas.

\end{itemize}

Só poderá existir \underline{\textbf{um usuário anônimo}} na base de dados. \textbf{Fiquem atentos!} Esse usuário anônimo será utilizado por todas as pessoas que realizarem denuncias anônimas, já que denúncias anônimas são registradas no sistema por um outra sistema computacional.

\section{Equipes}

As equipes devem ser formadas por grupos de alunos (mínimo 2 e máximo de 3 alunos). Cada grupo deve organizar-se de forma a que todos os seus membros contribuam na elaboração do trabalho (\underline{\textbf{Trabalho colaborativo!}}). Uma forma muito simples de organização é a criação de um repositório remoto (github, bitbucket, entre outros) onde cada membro pode fazer suas contribuições diretamente no repositório remoto criado.

\section{Avaliação}

Os trabalhos serão avaliados conforme os seguintes critérios:

\begin{itemize}
\itemsep 5mm

\item modelo de dados usando o diagrama de schema (50\%);

\item Implementação do sistema (30\%);

\item elaboração de documento (mínimo de 3 e máximo de 5 páginas) com a descrição do modelo de dados e da implementação, justificando as decisões tomadas durante a execução do projeto (10\%);

\item apresentação oral (por equipe) do modelo de dados e da implementação do sistema (10\%).

\end{itemize}

\section{Data de Entrega}

Os trabalhos precisam ser entregues pelas equipes até o dia \textbf{\underline{19/06/2018}}.

\end{document}