\documentclass[12pt]{article}

\title{Lista de Exercícios II\\
BAN-II}

\author{Professor: \textbf{Jeferson Souza (jefecomp)}}

\begin{document}

\date{}

\maketitle

\begin{center}
Data de Entrega: \textbf{\underline{15/03/2018}}
\end{center}

\section{Exercícios}

\begin{enumerate}
\itemsep 10mm

\item Especifique um diagrama de esquema que descreva o relacionamento de 5 entidades de uma sistema para controle de restaurantes.

\item O que é a linguagem de definição de dados (DDL)?

\item Escreva o comando SQL para criar um schema com o nome \textit{schema\_restaurante}.

\item Escreva os comandos SQL para criar todas as tabelas descritas na questão 1 dentro do schema \textit{schema\_restaurante}. Considere o uso de restrições nas especificações das tabelas; ex: não permitir que uma dada coluna seja inserida sem valor.

\item Considere o seguinte comando SQL: \textit{\textbf{CREATE TABLE} usuario(id bigserial primary key, nome varchar(40) \textbf{NOT NULL}, email varchar(50));}\\[-1mm]

Escreva o comando SQL para alterar a tabela usuario e adicionar a coluna cpf.

\item Escreva o comando SQL para criar uma tabela endereco e o comando SQL para criar uma tabela endereco\_usuario, onde a chave primária de endereco\_usuario será um identificador incrementado automaticamente. Na sequência, escreva o comando SQL para alterar a tabela endereco\_usuario, remover a coluna da chave primária, e adicionar as chaves estrangeiras das tabelas endereco e usuario como chave primária composta da tabela endereco\_usuario.

\item Crie uma visão para facilitar o acesso a dados das tabelas especificadas no diagrama de esquema da questão 1. 
\end{enumerate}


\end{document}