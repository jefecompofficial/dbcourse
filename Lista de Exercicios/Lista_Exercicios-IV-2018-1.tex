\documentclass[12pt]{article}

\title{Lista de Exercícios IV\\
BAN-II}

\author{Professor: \textbf{Jeferson Souza, MSc. (jefecomp)}}

\begin{document}

\date{}

\maketitle

\begin{center}
Data de Entrega: \textbf{\underline{05/04/2018}}
\end{center}

\section*{Exercícios}

\begin{enumerate}
\itemsep 10mm

\item Explique com as suas palavras qual a diferença entre funções e gatilhos.

\item Escreva uma função que ao ser executada permita apagar uma base de dados denominada banii.

\item Escreva uma função que concatene o nome e o sobrenome dos registros de uma tabela usuario, e retorne essa concatenação como resultado.

\item Escreva uma função de população de registros na base de dados que terá o seguinte comportamento:

\begin{enumerate}

\item A função deve criar uma tabela papel na base de dados com as seguintes colunas: id, descricao;

\item A função deve criar uma tabela usuario na base de dados com as seguintes colunas: id, nome, sobrenome, cpf, email, senha, papel;

\item A função deve inserir os seguintes papéis na tabela papel: administrador, gestor, e usuario;

\item O papel administrador tem poderes totais de manipulação da tabela de usuario;

\item O papel gestor pode inserir, atualizar, e consultar dados na tabela de usuario;

\item O papel usuario pode consultar dados na tabela usuario.

\item A função deve inserir, pelo menos, um usuario de cada tipo (administrador, gestor, usuario) na tabela usuario.

\end{enumerate}

\item Escreva uma função que retorne um tipo composto com o nome, cpf, e email de cada um dos usuarios da tabela usuario criada na questão anterior.

\item Crie um gatilho que ao inserir um registro na tabela de usuario dispare uma função que cadastre uma senha gerada automaticamente. A senha deve consistir na concatenação do nome e do cpf do usuario.

\item Crie um gatilho que ao remover um registro da tabela usuario dispare uma função que insira um registro na tabela auditoria\_usuario, com as seguintes colunas: id, nome, sobrenome, cpf, email, senha, papel, data\_remocao.

\end{enumerate}

\end{document}